% REVISÃO BIBLIOGRÁFICA

\chapter{\normalsize{FUNDAMENTAÇÃO TEÓRICA}} \label{cap:fundamentação_teorica}

\thispagestyle{empty}

A fundamentação teórica tem como objetivo apresentar e discutir os principais
conceitos, definições e abordagens encontrados na literatura científica que
servem de base para o desenvolvimento deste trabalho.

Inicialmente, são abordados os conceitos fundamentais relacionados ao tema da
pesquisa, permitindo contextualizar o problema estudado e estabelecer uma
base conceitual sólida. Em seguida, são discutidos os principais modelos,
métodos e abordagens propostos por diferentes autores, destacando suas
vantagens, limitações e campos de aplicação.

A revisão da literatura possibilita identificar lacunas existentes e
justificar a relevância da pesquisa desenvolvida, além de fornecer subsídios
para a definição da metodologia adotada e para a análise crítica dos
resultados obtidos.

Caso necessário a fundamentação teórica pode ser dividia em mais capítulos.

Para fazer uma nota de rodapé faça dessa maneira \footnote[1]{Texto da nota de rodapé}. 


\section{Estrutura de Trabalhos Acadêmicos}

A numeração progressiva das seções deve seguir a NBR 6024, permitindo a
organização hierárquica do conteúdo. Um exemplo de referência cruzada pode
ser observado no Capítulo~\ref{cap:metodologia}.

\subsection{Uso de Siglas}

Na primeira ocorrência, a sigla deve ser precedida de sua forma por extenso,
como Associação Brasileira de Normas Técnicas (ABNT). Nas ocorrências
seguintes, utiliza-se apenas a sigla.



\section{Inserindo Equações}

A seguir tem um exemplo de como inserir equações no texto. A Equação \ref{eq:modelo} está descrita a seguir.

\begin{equation}
	M \ddot{x}(t) + C \dot{x}(t) + K x(t) = F(t)
	\label{eq:modelo}
\end{equation}

\noindent em que $[M]$ é a matriz de massa, $[C]$ é a matriz de amortecimento, $[K]$ é a matriz de rigidez, $\{x(t)\}$ representa o vetor de deslocamentos, $\{\dot{x}(t)\}$ e $\{\ddot{x}(t)\}$ são as derivadas de primeira e segunda ordem, $\{F(t)\}$ é o vetor de forças externas.


Para fazer sequência de equações pode ser feito como apresentado nas Eq.~(\ref{eq:A}) e (\ref{eq:B})

\begin{gather}
	\oint H \cdot ds=ni
	\label{eq:A} \\
	l_{fe} \cdot H_{fe}+2x \cdot H_{x}=ni
	\label{eq:B}
\end{gather}


\section{Tabelas}

A apresetnação de dados pode ser feito conforme a  Tab.~\ref{tab:parametros}.

\begin{table}[H]
	\centering
	
	\begin{threeparttable}
		\caption{Modelo coluna e repetições}
		\label{tab:parametros}
		
		
		\begin{tabular}{cccc}
			\hline
			Tratamento 1 & Tratamento 2 & Tratamento 3 & Tratamento 4 \\
			\hline
			123 & 4512 & 234 & 807 \\
			778 & 5678 & 543 & 755 \\
			409 & 7856 & 465 & 265 \\
			498 & 8657 & 584 & 646 \\
			321 & 4535 & 445 & 343 \\
			456 & 4666 & 243 & 966 \\
			\hline
		\end{tabular}
		
		% O comando \begin{tablenotes} faz o alinhamento automático
			\begin{tablenotes}[para, flushleft]
				\small
				\item Fonte: Elaborado pelo Autor
			\end{tablenotes}
			
		\end{threeparttable}
\end{table}

A Tabela~\ref{tab:multicolumn_multirow} apresenta um exemplo do uso de multicoluna e multilinha.

\begin{table}[H]
	\centering
	
	\begin{threeparttable}
		\caption{Resultados experimentais por grupo e tratamento}
		\label{tab:multicolumn_multirow}
		
		\begin{tabular}{ccccc}
			\hline
			\multirow{2}{*}{Grupo} 
			& \multicolumn{4}{c}{Tratamentos} \\
			\cline{2-5}
			& Tratamento 1 & Tratamento 2 & Tratamento 3 & Tratamento 4 \\
			\hline
			\multirow{3}{*}{A} 
			& 123 & 4512 & 234 & 807 \\
			& 778 & 5678 & 543 & 755 \\
			& 409 & 7856 & 465 & 265 \\
			\hline
			\multirow{3}{*}{B} 
			& 498 & 8657 & 584 & 646 \\
			& 321 & 4535 & 445 & 343 \\
			& 456 & 4666 & 243 & 966 \\
			\hline
		\end{tabular}
		
		\begin{tablenotes}[para, flushleft]
			\small
			\item Fonte: Elaborado pelo autor.
		\end{tablenotes}
		
	\end{threeparttable}
\end{table}

\newpage

\section{Figuras}

A Figura~\ref{fig:exemplo} ilustra como deve ser colocada uma figura no trabalho.

\begin{figure}[H]
	\centering
	\caption{Legenda da Figura}
	\label{fig:exemplo}
	
	\begin{minipage}{0.5\textwidth}
		\centering
		\includegraphics[width=\linewidth]{figuras/figura_exemplo.jpg}
		\par\vspace{0.3em}
		\raggedright
		{\small Fonte: Elaborado pelo autor.}
	\end{minipage}
\end{figure}

%\newpage

Abaixo está um exemplo de subfiguras \ref{subfig:1} e \ref{subfig:2} presentes na Fig.~\ref{subfigure}.

\begin{comment}

\begin{figure}[H]
	\caption{Subfigruas}
%	\label{subfigure}
	\begin{subfigure}{0.45\textwidth}
		\centering
		\includegraphics[width=\textwidth]{figuras/subfigure1.png}
		\caption{Legenda subfigura A.}
%		\label{subfig:1}
	\end{subfigure}
	\hspace{10pt}
	\begin{subfigure}{0.45\textwidth}
		\centering
		\includegraphics[width=\textwidth]{figuras/subfigure2.png}
		\caption{Legenda subfigrua B.}
%		\label{subfig:2}
	\end{subfigure}
\end{figure}
\end{comment}


\begin{figure}[H]
	\centering
	\caption{Subfiguras}
	\label{subfigure}
	
	\begin{minipage}{\textwidth}
		\centering
		
		\begin{subfigure}{0.45\textwidth}
			\centering
			\caption{Legenda subfigura A.}
			\label{subfig:1}
			\includegraphics[width=\linewidth]{figuras/subfigure1.png}
		\end{subfigure}
		\hspace{10pt}
		\begin{subfigure}{0.45\textwidth}
			\centering
			\caption{Legenda subfigura B.}
			\label{subfig:2}
			\includegraphics[width=\linewidth]{figuras/subfigure2.png}
		\end{subfigure}
		
		\par\vspace{0.5em}
		\raggedright
		{\small Fonte: Elaborado pelo autor.}
	\end{minipage}
\end{figure}

